\documentclass{article}
\usepackage[utf8]{inputenc}
\usepackage{graphicx}
\usepackage{url}
%\usepackage[left=1.5cm,right=1.5cm,top=2cm,bottom=2cm]{geometry}
\title{LSINF2345 - Course project \\ A Simulated Bike Race}
\author{Alexandre Hauet \& Florian Thuin}
\date{\today} 

\def\blurb{\textsc{Université catholique de Louvain\\
  École polytechnique de Louvain\\
  Pôle d'ingénierie informatique}}
\def\clap#1{\hbox to 0pt{\hss #1\hss}}%
\def\ligne#1{%
  \hbox to \hsize{%
    \vbox{\centering #1}}}%
\def\haut#1#2#3{%
  \hbox to \hsize{%
    \rlap{\vtop{\raggedright #1}}%
    \hss
    \clap{\vbox{\vfill\centering #2\vfill}}%
    \hss
    \llap{\vtop{\raggedleft #3}}}}%


\begin{document}

\begin{titlepage}
\thispagestyle{empty}\vbox to 1\vsize{%
  \vss
  \vbox to 1\vsize{%
    \haut{\raisebox{-5mm}{\includegraphics[width=2.5cm]{logo_ucl.pdf}}}{\blurb}{\raisebox{-5mm}{\includegraphics[scale=0.20]{ingi_logo.png}}}
    \vfill
    \ligne{\Huge \textbf{\textsc{LSINF2345}}}
     \vspace{5mm}
    \ligne{\huge \textbf{\textsc{Course project }}}
     \vspace{15mm}
    \ligne{\Large \textbf{\textsc{A Simulated Bike Race}}}
    \vspace{5mm}
    \ligne{\large{\textsc{\today}}}
    \vfill
    \vspace{5mm}
    \ligne{%
         \textsc{Group 10\\Alexandre Hauet\\Florian Thuin} 
      }
      \vspace{5mm}
    }%
  \vss
  }
\end{titlepage}






\section*{Introduction}
For this assignment, we must resolve three problems linked to a game \textbf{Bike Race}.  To solve this problem, 
we realize a distributed system in Erlang. Specifically, we use Riak : a distributed NoSQL key-value data store
 that offers high availability, fault tolerance, operational simplicity, and scalability\footnote{\url{https://en.wikipedia.org/wiki/Riak}}.
 Riak makes our life easier, because it manages the send of messages for us.


\section*{Problem 0}
For this first problem of the assignment, we need to start by setting the track, where no biker fails and no special
consideration has been given to broadcasting messages. We use the best-effort broadcast (beb) to implement 
this problem.\newline


At the begin of the game, each biker will have an id.  To implement the best-effort broadcast, we take advantage of
Riak. So each nodes during a round will store its decision in the key store. Others will reach the decision using the id of 
the biker.\newline

We can see that the properties of best-effort broadcast aren't violate. The message are not duplicated, they are written
only once times in the key store by the bikers.


\section*{Problem 1}

\section*{Problem 2}
For this problem, we must handle the failure cause by the fact that a biker may fall down during the race.
To solve this problem, we need to modify our solution of problem 1. Our design requires the use of a 
failure detector. To build it, we use a simple heartbeat algorithm.\newline

A heartbeat is a mechanism can detect a crash node within a bounded time. The algorithm exchanges 
messages all processes and uses a specific timeout mechanism initialized with a delay. During this period,
all processes have the time to send messages (request messages and reply messages). After they are considered 
as crashed. But we must be careful that all processes have enough time to send the messages. 

\end{document}
